% Reference commands
\newcommand{\ch}[1]{\ref{ch:#1}}
\newcommand{\sect}[1]{\ref{sec:#1}}
\newcommand{\fig}[1]{\ref{fig:#1}}

% Math commands
\newcommand{\dcov}[1][\mu]{\partial_{#1}}
\newcommand{\dcon}[1][\mu]{\partial^{#1}}
\newcommand{\lag}{\mathcal{L}}
\newcommand{\sd}{\slashed{\partial}}

% Simple command for figures
\newcommand{\ownFigure}[5]
{
\begin{figure}[#1] %  figure placement: here, top, bottom, or page
   \centering
   \includegraphics[width=#2\textwidth]{images/#3} 
   \caption{#5}
   \label{fig:#4}
\end{figure}
}

%Feynman diagram commands
\newcommand{\simpleFeynman}[5]{
\begin{figure}[#1]
    \centering
    \feynmandiagram [#2] {
      #3
    };
    \caption{#5}
    \label{fig:#4}
\end{figure}
}

\newcommand{\manualFeynman}[4]{
\begin{figure}[#1]
    \centering
    \begin{tikzpicture}
      \begin{feynman}
        #2
      \end{feynman}
    \end{tikzpicture}
    \caption{#4}
    \label{fig:#3}
\end{figure}
}